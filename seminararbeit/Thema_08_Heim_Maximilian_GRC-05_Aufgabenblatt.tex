\documentclass[11pt]{article}
\usepackage[ngerman]{babel}
\usepackage{float}
\usepackage{graphicx}
\usepackage{hyperref}
\usepackage{cleveref}
\author{Maximilian Heim}
\crefname{section}{Kapitel}{Kapitel}
\crefname{figure}{Abbildung}{Abbildung}
\title{Identitäts- und Berechtigungsmanagement \\ \large{}Aufgabenblatt mit Lösungen}
\begin{document}
\maketitle
\section*{Aufgabe 1: SSO}
Was sind die Vorteile und Anwendungsbereiche von Single Sign On?

\section*{Aufgabe 2: Rechtliche Aspekte IAM vs. CIAM}
Wie unterscheiden sich die Rechtlichen Anforderungen im Rahmen des IAM's von denen des CIAM's?
\subsection{Antwort}
Im Rahmen des internen Identitäts- und Berechtigungsmanagements spezifizieren z.B. KonTraG, GoBD, EuroSOX und EU-DSVGO/BDSG Anforderungen an das Identitäts- und Berechtigungsmanagement. Im Rahmen des CIAM's spezifiziert die EU-DSGVO und das BDSG sehr zentrale Anforderungen.

\section*{Aufgabe 3: Organisatorische Rollen}
Welche organisatorischen Rollen können im Identitäts- und Berechtigungsmanagement differenziert werden?
\subsection{Antwort}

\section*{Aufgabe 4: Digitale Identität}
Was ist eine digitale Identität? Woraus setzt sie sich zusammen?
\subsection{Antwort}

\end{document}