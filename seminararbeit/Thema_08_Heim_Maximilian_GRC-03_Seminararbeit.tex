\documentclass[10pt]{article}
\usepackage[ngerman]{babel}
\usepackage[%
  backend=biber,
  style=alphabetic,
  sorting=ynt]{biblatex}
\addbibresource{bib.bib}
\author{Maximilian Heim}
\title{Identitäts und Berechtigungsmanagement}
\begin{document}
\maketitle
\newpage
\tableofcontents
\newpage
\section{Einleitung}
\subsection{Aufgabenstellung}
\subsection{Forschungsfragen}
\section{Grundlagen}
\subsection{Einordnung}
\subsection{Identität}
\subsection{Identitätsmanagement}
\subsection{Berechtigung}
\subsection{Berechtigungsmanagement}
\subsection{Erkenntnisse im Kontext von IT-GRC}
\section{Methoden, Technologien und Tools}
\subsection{Betriebliche Motivaiton}
\subsection{Standards}
\subsection{Methoden und Prozesse}
\subsection{Technologien und Tools}
\subsection{Erkenntnisse im Kontext von IT-GRC}
\section{Betriebliches Identitäts- und Berechtigungsmanagement}
\subsection{Überblick}
\subsection{Organisatorische Aspekte}
\subsection{Technische Aspekte}
\subsection{Wirtschaftliche Aspekte}
\subsection{Erkenntnisse im Kontext von IT-GRC}
\section{Fazit}
\subsection{Zusammenfassung}
\subsection{Beantwortung der Forschungsfragen}
\section{Eidesstattliche Versicherung}
\section{Literaturverzeichnis}
\section{Quellenverzeichnis}
\printbibliography
\end{document}