\documentclass[10pt]{article}
\usepackage[ngerman]{babel}
\usepackage[%
  backend=biber,
  style=alphabetic,
  sorting=ynt]{biblatex}
\addbibresource{bib.bib}
\author{Maximilian Heim}
\title{Identitäts und Berechtigungsmanagement}
\begin{document}
\maketitle
\newpage
\tableofcontents
\newpage
\section{Einleitung}
\subsection{Aufgabenstellung}
\subsection{Forschungsfragen}
\section{Grundlagen}
\subsection{Einordnung}
\subsection{Identität}
Um den Begriff Identitätsmanagement zu definieren sollte zuerst der Begriff der Identität definiert werden. In der Philosophie wird Identität über die Ununterscheidbarkeit von Dingen definiert. Nach dem Identitätsprinzip sind zwei Dinge genau dann identisch wenn sich zwischen ihnen keine Unterschiede finden lassen.
\subsection{Identitätsmanagement}
Identitätsmanagement beschreibt Prozesse mit denen die Attribute einer Einheit verwaltet werden. Grundlegend lassen sich hier verschiedene Themen unterscheiden, so z.B.
\begin{itemize}
  \item Lebenszyklus von Identitäten
  \item
\end{itemize}
\subsection{Berechtigung}
\subsection{Berechtigungsmanagement}
\subsection{Erkenntnisse im Kontext von IT-GRC}
\section{Methoden, Technologien und Tools}
\subsection{Betriebliche Motivation}
\subsection{Standards}
\paragraph{BSI}
Das Bundesamt für Sicherheit der Informationstechnik (BSI) definiert mit BSI-Standard 200-3 einen Leitfaden zur Risikobewertung. In BSI-Standard 200-1 werden Sicherheitsmaßnahmen definiert die zur Behandlung der Risiken geeignet sind. In Bezug auf den BSI-Standard definiert das IT-Grundschutz-Kompendium des BSI's Prozessbausteine zur Umsetzung des ISMS. Hier wird im Prozessbaustein "ORP.4 Identitäts- und Berechtigungsmanagent" auf verschiedene Anforderungen für die Umsetzung von IAM eingegangen. Kapitel 3.1 definiert Basis-Anforderungen welche umgesetzt werden müssen. Kapitel 3.2 definiert Standard-Anforderungen welche umgesetzt werden sollten. Kapitel 3.3 definiert Anforderungen welche bei erhöhtem Schutzbedarf umgesetzt werden sollten. Das IT-Grundschuztz-Kompendium definiert zusätzlich zu ORP.4 System-Bausteine wie SYS.1.3 und APP.2.1. Diese enthalten konkrete Maßnahmen zur Umsetzung des IAM für System-Komponenten wie Betriebssysteme und Verzeichnisdienste.
\paragraph{ISO 27001 Annex A.9}
ISO 27001 definiert mit Anhang A.9 die Zugangssteuerung.
\paragraph{SO/IEC 29146}
\paragraph{NIST 800-53A}
\subsection{Methoden und Prozesse}
\subsection{Technologien und Tools}
\subsection{Erkenntnisse im Kontext von IT-GRC}
\section{Betriebliches Identitäts- und Berechtigungsmanagement}
\subsection{Überblick}
\subsection{Organisatorische Aspekte}
\subsection{Technische Aspekte}
\subsection{Wirtschaftliche Aspekte}
\subsection{Erkenntnisse im Kontext von IT-GRC}
\section{Fazit}
\subsection{Zusammenfassung}
\subsection{Beantwortung der Forschungsfragen}
\section{Eidesstattliche Versicherung}
\section{Literaturverzeichnis}
\section{Quellenverzeichnis}
\printbibliography
\end{document}